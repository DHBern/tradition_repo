%----------------------------------------------------------------------------------------
%	CHAPTER 10
%----------------------------------------------------------------------------------------

\chapter{Services}
\section{Baseresource: }
http://localhost:8080/
\section{/stemma}
\textbf{Method Name: }getAllStemmata \\ Gets a list of all Stemmata available, as dot format
\begin{get}
/getallstemmata/fromtradition/\{tradId\}
\end{get}
\begin{response}
list of stemmata as dot
\end{response}
\begin{parameter}
tradId as string
\end{parameter}
\textbf{Method Name: }setStemma \\ Puts the Stemma of a DOT file in the database
\begin{post}
/newstemma/intradition/\{tradId\}
\end{post}
\begin{request}
 as application/json
\end{request}
\begin{response}
stemma as dot
\end{response}
\begin{parameter}
tradId as string
\end{parameter}
\textbf{Method Name: }reorientStemma \\ Reorients a stemma tree with a given new root node
\begin{post}
/reorientstemma/fromtradition/\{tradId\}/withtitle/\{stemmaTitle\}/withnewrootnode/\{nodeId\}
\end{post}
\begin{response}
stemma as dot
\end{response}
\begin{parameter}
tradId as string
\end{parameter}
\begin{parameter}
stemmaTitle as string
\end{parameter}
\begin{parameter}
nodeId as string
\end{parameter}
\textbf{Method Name: }getStemma \\ Returns JSON string with a Stemma of a tradition in DOT format
\begin{get}
/getstemma/fromtradition/\{tradId\}/withtitle/\{stemmaTitle\}
\end{get}
\begin{response}
stemma as dot
\end{response}
\begin{parameter}
tradId as string
\end{parameter}
\begin{parameter}
stemmaTitle as string
\end{parameter}
\section{/relation}
\textbf{Method Name: }delete \\ Remove all relationships, as it is done in https://github.com/tla/stemmaweb/blob/master/lib/stemmaweb/Controller/Relation.pm line 271) in Relationships of type RELATIONSHIP between the two nodes.
\begin{post}
/deleterelationship/fromtradition/\{tradId\}
\end{post}
\begin{request}
relationshipModel as application/json
\end{request}
\begin{response}
 as text/plain: HTTP Response 404 when no node was found, 200 When relationships where removed
\end{response}
\begin{parameter}
tradId as string
\end{parameter}
\textbf{Method Name: }create \\ Creates a new relationship between the two nodes specified.
\begin{post}
/createrelationship
\end{post}
\begin{request}
relationshipModel as application/json
\end{request}
\begin{response}
 as application/json
\end{response}
\textbf{Method Name: }getAllRelationships \\ Get a list of all relationships from a given tradition.
\begin{get}
/getallrelationships/fromtradition/\{tradId\}
\end{get}
\begin{response}
list of relationshipModel as application/json
\end{response}
\begin{parameter}
tradId as string
\end{parameter}
\textbf{Method Name: }deleteById \\ Removes a relationship by ID.
\begin{delete}
/deleterelationshipbyid/withrelationship/\{relationshipId\}
\end{delete}
\begin{response}
 as text/plain: HTTP Response 404 when no Relationship was found with id, 200 and a model of the relationship in JSON when the Relationship was removed
\end{response}
\begin{parameter}
relationshipId as string
\end{parameter}
\section{/tradition}
\textbf{Method Name: }getAllRelationships \\ Gets a list of all relationships of a tradition with the given id.
\begin{get}
/getallrelationships/fromtradition/\{tradId\}
\end{get}
\begin{response}
 as application/json: Http Response 200 and a list of relationship model in JSON
\end{response}
\begin{parameter}
tradId as string
\end{parameter}
\textbf{Method Name: }changeTraditionMetadata \\ Changes the metadata of the tradition.
\begin{post}
/changemetadata/fromtradition/\{tradId\}
\end{post}
\begin{request}
traditionModel as application/json
\end{request}
\begin{response}
 as application/json: OK and information about the tradition in JSON on success or an ERROR in JSON format
\end{response}
\begin{parameter}
tradId as string
\end{parameter}
\textbf{Method Name: }getAllTraditions \\ Gets a list of all the complete traditions in the database.
\begin{get}
/getalltraditions
\end{get}
\begin{response}
 as application/json: Http Response 200 and a list of tradition models in JSON on success or Http Response 500
\end{response}
\textbf{Method Name: }getAllWitnesses \\ Gets a list of all the witnesses of a tradition with the given id.
\begin{get}
/getallwitnesses/fromtradition/\{tradId\}
\end{get}
\begin{response}
 as application/json: Http Response 200 and a list of witness models in JSON on success or an ERROR in JSON format
\end{response}
\begin{parameter}
tradId as string
\end{parameter}
\textbf{Method Name: }getTradition \\ Returns GraphML file from specified tradition owned by user
\begin{get}
/gettradition/withid/\{tradId\}
\end{get}
\begin{response}
 as application/json: XML data
\end{response}
\begin{parameter}
tradId as string
\end{parameter}
\textbf{Method Name: }deleteTraditionById \\ Removes a complete tradition
\begin{delete}
/deletetradition/withid/\{tradId\}
\end{delete}
\begin{response}
 as text/plain: http response
\end{response}
\begin{parameter}
tradId as string
\end{parameter}
\textbf{Method Name: }importGraphMl \\ Imports a tradition by given GraphML file and meta data
\begin{post}
//newtraditionwithgraphml
\end{post}
\begin{request}
 as multipart/form-data
\end{request}
\begin{response}
 as application/json: Http Response with the id of the imported tradition on success or an ERROR in JSON format
\end{response}
\textbf{Method Name: }getDot \\ Returns DOT file from specified tradition owned by user
\begin{get}
/getdot/fromtradition/\{tradId\}
\end{get}
\begin{response}
 as application/json: XML data
\end{response}
\begin{parameter}
tradId as string
\end{parameter}
\section{/reading}
\textbf{Method Name: }changeReadingProperties \\ Changes properties of a reading according to its keys
\begin{post}
/changeproperties/ofreading/\{readId\}
\end{post}
\begin{request}
 as application/json
\end{request}
\begin{response}
readingModel as application/json
\end{response}
\begin{parameter}
readId as long
\end{parameter}
\textbf{Method Name: }getReading \\ Returns a single reading by global neo4j id
\begin{get}
/getreading/withreadingid/\{readId\}
\end{get}
\begin{response}
readingModel as application/json
\end{response}
\begin{parameter}
readId as long
\end{parameter}
\textbf{Method Name: }duplicateReading \\ Duplicates a reading in a specific tradition. Opposite of merge
\begin{post}
/duplicatereading
\end{post}
\begin{request}
duplicateModel as application/json
\end{request}
\begin{response}
GraphModel as application/json
\end{response}
\textbf{Method Name: }mergeReadings \\ Merges two readings into one single reading in a specific tradition. Opposite of duplicate
\begin{post}
/mergereadings/first/\{firstReadId\}/second/\{secondReadId\}
\end{post}
\begin{response}
 as application/json: Status.OK on success or Status.INTERNAL\_SERVER\_ERROR with a detailed message.
\end{response}
\begin{parameter}
secondReadId as long
\end{parameter}
\begin{parameter}
firstReadId as long
\end{parameter}
\textbf{Method Name: }splitReading \\ Splits up a single reading into several ones in a specific tradition. Opposite of compress
\begin{post}
/splitreading/ofreading/\{readId\}/withsplitindex/\{splitIndex\}
\end{post}
\begin{request}
 as text/plain
\end{request}
\begin{response}
GraphModel as application/json
\end{response}
\begin{parameter}
readId as long
\end{parameter}
\begin{parameter}
splitIndex as int
\end{parameter}
\textbf{Method Name: }getNextReadingInWitness \\ gets the next readings from a given readings in the same witness
\begin{get}
/getnextreading/fromwitness/\{witnessId\}/ofreading/\{readId\}
\end{get}
\begin{response}
readingModel as application/json
\end{response}
\begin{parameter}
readId as long
\end{parameter}
\begin{parameter}
witnessId as string
\end{parameter}
\textbf{Method Name: }getPreviousReadingInWitness \\ gets the previous readings from a given readings in the same witness
\begin{get}
/getpreviousreading/fromwitness/\{witnessId\}/ofreading/\{readId\}
\end{get}
\begin{response}
readingModel as application/json
\end{response}
\begin{parameter}
readId as long
\end{parameter}
\begin{parameter}
witnessId as string
\end{parameter}
\textbf{Method Name: }getAllReadings \\ Returns a list of all readings in a tradition
\begin{get}
/getallreadings/fromtradition/\{tradId\}
\end{get}
\begin{response}
list of readingModels as application/json
\end{response}
\begin{parameter}
tradId as string
\end{parameter}
\textbf{Method Name: }getIdenticalReadings \\ Get all readings which have the same text and the same rank between given ranks
\begin{get}
/getidenticalreadings/fromtradition/\{tradId\}/fromstartrank/\{startRank\}/toendrank/\{endRank\}
\end{get}
\begin{response}
list of list of readingModels as application/json
\end{response}
\begin{parameter}
endRank as long
\end{parameter}
\begin{parameter}
tradId as string
\end{parameter}
\begin{parameter}
startRank as long
\end{parameter}
\textbf{Method Name: }getCouldBeIdenticalReadings \\ Returns a list of a list of readingModels with could be one the same rank without problems
\begin{get}
/couldbeidenticalreadings/fromtradition/\{tradId\}/fromstartrank/\{startRank\}/toendrank/\{endRank\}
\end{get}
\begin{response}
list of readingModels as application/json
\end{response}
\begin{parameter}
endRank as long
\end{parameter}
\begin{parameter}
tradId as string
\end{parameter}
\begin{parameter}
startRank as long
\end{parameter}
\textbf{Method Name: }compressReadings \\ Compress two readings into one. Texts will be concatenated together (with or without a space or extra text. The reading with the lower rank will be given first. Opposite of split
\begin{post}
/compressreadings/read1id/\{read1Id\}/read2id/\{read2Id\}/concatenate/\{con\}
\end{post}
\begin{request}
 as text/plain
\end{request}
\begin{response}
 as application/json: status.ok if compress was successful. Status.INTERNAL\_SERVER\_ERROR with a detailed message if not concatenated
\end{response}
\begin{parameter}
read1Id as long
\end{parameter}
\begin{parameter}
read2Id as long
\end{parameter}
\begin{parameter}
con as string
\end{parameter}
\section{/witness}
\textbf{Method Name: }getWitnessAsText \\ finds a witness in the database and returns it as a string
\begin{get}
/gettext/fromtradition/\{tradId\}/ofwitness/\{witnessId\}
\end{get}
\begin{response}
a witness as a string
\end{response}
\begin{parameter}
tradId as string
\end{parameter}
\begin{parameter}
witnessId as string
\end{parameter}
\textbf{Method Name: }getWitnessAsTextBetweenRanks \\ find a requested witness in the data base and return it as a string according to define start and end readings (including the readings in those ranks). if end-rank is too high or start-rank too low will return till the end/from the start of the witness
\begin{get}
/gettext/fromtradition/\{tradId\}/ofwitness/\{witnessId\}/fromstartrank/\{startRank\}/toendrank/\{endRank\}
\end{get}
\begin{response}
a witness as a string
\end{response}
\begin{parameter}
endRank as string
\end{parameter}
\begin{parameter}
tradId as string
\end{parameter}
\begin{parameter}
startRank as string
\end{parameter}
\begin{parameter}
witnessId as string
\end{parameter}
\textbf{Method Name: }getWitnessAsReadings \\ finds a witness in the database and returns it as a list of readings
\begin{get}
/getreadinglist/fromtradition/\{tradId\}/ofwitness/\{witnessId\}
\end{get}
\begin{response}
list of readingModels as application/json
\end{response}
\begin{parameter}
tradId as string
\end{parameter}
\begin{parameter}
witnessId as string
\end{parameter}
\section{/user}
\textbf{Method Name: }create \\ Creates a user based on the parameters submitted in JSON.
\begin{post}
/createuser
\end{post}
\begin{request}
userModel as application/json
\end{request}
\begin{response}
userModel as application/json
\end{response}
\textbf{Method Name: }getUserById \\ Gets a user by the id.
\begin{get}
/getuser/withid/\{userId\}
\end{get}
\begin{response}
userModel as application/json
\end{response}
\begin{parameter}
userId as string
\end{parameter}
\textbf{Method Name: }deleteUserById \\ Removes a user and all his traditions
\begin{delete}
/deleteuser/withid/\{userId\}
\end{delete}
\begin{response}
 as text/plain: OK on success or an ERROR in JSON format
\end{response}
\begin{parameter}
userId as string
\end{parameter}
\textbf{Method Name: }getTraditionsByUserId \\ Get all Traditions of a user
\begin{get}
/gettraditions/ofuser/\{userId\}
\end{get}
\begin{response}
 as application/json: OK on success or an ERROR in JSON format
\end{response}
\begin{parameter}
userId as string
\end{parameter}


%----------------------------------------------------------------------------------------
%	CHAPTER 11
%----------------------------------------------------------------------------------------

\chapter{Models}

\section{relationshipModel}
\begin{property}
a\_derivable\_from\_b as string
\end{property}
\begin{property}
alters\_meaning as string
\end{property}
\begin{property}
annotation as string
\end{property}
\begin{property}
b\_derivable\_from\_a as string
\end{property}
\begin{property}
displayform as string
\end{property}
\begin{property}
extra as string
\end{property}
\begin{property}
id as string
\end{property}
\begin{property}
is\_significant as string
\end{property}
\begin{property}
non\_independent as string
\end{property}
\begin{property}
reading\_a as string
\end{property}
\begin{property}
reading\_b as string
\end{property}
\begin{property}
scope as string
\end{property}
\begin{property}
source as string
\end{property}
\begin{property}
target as string
\end{property}
\begin{property}
type as string
\end{property}
\begin{property}
witness as string
\end{property}
\section{readingModel}
\begin{property}
grammar\_invalid as string
\end{property}
\begin{property}
id as string
\end{property}
\begin{property}
is\_common as string
\end{property}
\begin{property}
is\_end as string
\end{property}
\begin{property}
is\_lacuna as string
\end{property}
\begin{property}
is\_lemma as string
\end{property}
\begin{property}
is\_nonsense as string
\end{property}
\begin{property}
is\_ph as string
\end{property}
\begin{property}
is\_start as string
\end{property}
\begin{property}
join\_next as string
\end{property}
\begin{property}
join\_prior as string
\end{property}
\begin{property}
language as string
\end{property}
\begin{property}
lexemes as string
\end{property}
\begin{property}
normal\_form as string
\end{property}
\begin{property}
rank as long
\end{property}
\begin{property}
text as string
\end{property}
\section{traditionModel}
\begin{property}
id as string
\end{property}
\begin{property}
isPublic as string
\end{property}
\begin{property}
language as string
\end{property}
\begin{property}
name as string
\end{property}
\begin{property}
ownerId as string
\end{property}
\section{duplicateModel}
\begin{property}
readings as long
\end{property}
\begin{property}
witnesses as string
\end{property}
\section{graphModel}
\begin{property}
readings as long
\end{property}
\begin{property}
witnesses as string
\end{property}
\section{userModel}
\begin{property}
id as string
\end{property}
\begin{property}
isAdmin as string
\end{property}
